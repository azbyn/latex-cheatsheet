\section{Environments}

\subsection{Text alignment}

%\begin{center} \code{\begin{center} ... \end{center}}  \end{center}
%\begin{flushleft} \code{\begin{flushleft} ... \end{flushleft}} \end{flushleft}
%\begin{flushright} \code{\begin{flushright} ... \end{flushright}} \end{flushright}

\code{\begin{center} Centered \end{center}} \\
\code{\begin{flushleft} Left \end{flushleft}} \\
\code{\begin{flushright} Right \end{flushright}}

\subsection{Mini Page} %http://www.sascha-frank.com/latex-minipage.html
Useful for putting things side by side
\subsubsection{Usage}
\begin{latex}
\begin{minipage}[adjusting]{width of the minipage}
 Text | Images | ... 
\end{minipage} 
\end{latex}
\subsubsection{Adjustment}
When adjusting the choices is: \coden{c} (centers), \coden{t} (top) and \coden{b} (bottom).
By default, \coden{c} is used for centering. It is aligned by \code{t} and/or \code{b}
at the highest (top line) and/or at the lowest line (bottom line).
\subsubsection{Further options}
Besides there are still further options, which however in practical application the minipage does not play a role like the height and the adjustment (again \coden{c}, \coden{t} and \coden{b}) within the minipage.

Example of further options:
\code{\begin{minipage}[t][5cm][b]{0,5\textwidth}}

  This minipage now has a defined height of 5cm, and the content will now be aligned to the bottom of the minipage.
\subsubsection{Hint}

A mistake that is often made is, there is a blank line between the \code{\end{minipage}} and
\code{\begin{minipage}} left. Then the pages are no longer together.
  
\subsubsection{Example}
\begin{latex}
\begin{minipage}[t]{0.3\textwidth}
  \includegraphics[width=\textwidth]{pic1}
\end{minipage}

\end{latex}

\subsection{Others}
For \coden{abstract}, \coden{titlepage}, \coden{thebibliography}, see
\nameref{docStructure} (page \pageref{docStructure}).

For math stuff, see \nameref{math}\footnote{Natürlich}.
