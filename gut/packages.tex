\section{Packages}
\subsection{Math}
\begin{latex}
\usepackage{amsmath} % adds a lot of math stuff 
\usepackage{mathtools} % fixes some amsmath things and adds more
\usepackage{amssymb} % adds a lot of math symbols 
\usepackage{amsfonts} % stuff like \mathbb
\usepackage{mathrsfs} % for \mathscr
\end{latex}

\subsection{Colors}
\begin{latex}
\usepackage{color}
%or even better (adds names like \RoyalPurple)
\usepackage[dvipsnames]{xcolor}
\end{latex}
\subsubsection* {Defining colors}
\code{\definecolor{name}{HTML}{RRGGBB}}, we can could also use \code{rgb} or \code{gray} instead of
\code{HTML}.

The option \texttt{[dvipsnames]} from \code{xcolor} defines PascalCase color names like in css like
'\code{MidnightBlue}'.

\subsubsection{Using colors}
\codeshow{\colorbox{blue}{fundal}}\\
\codeshow{\fcolorbox{Black}{White}{text}} - \code{\fcolorbox{margine}{fundal}{text}}\\
\code{\pagecolor{White}}\\
or \codeshow{\color{red} hello \color{black}}\\
or \codeshow{{\color{red} hello}}

\subsection{Images}
\begin{latex}
%preable
\usepackage{graphicx}
\usepackage{epstopdf}% for eps images

%in document
\includegraphics[scale=.4, angle=45]{something}
\includegraphics[width=3cm, height=4cm]{something}
\includegraphics[width=0.5\textwidth]{something}
\end{latex}
Tell where the images are, relative to main \texttt{.tex} file:\\
\code{\graphicspath}\texttt{\{ \{./images/\} \{./images2/\} \}}

\subsection{Figure}\label{figure}
\begin{latex}
\begin{figure}[!hbt]
  \includegraphics[width=\textwidth]{plot}
  \caption{Caption}\label{plot}
\centering
\end{figure}
\end{latex}
\note{You can use \code{wrapfigure} for wraping text around a figure.
  See \nameref{wraptable}} (page \pageref{wraptable}).
Options:
\begin{longtable}{l l}
  \justexplain{h}{place here - kinda}
  \justexplain{h!}{place here - more strict}
  \justexplain{t}{top of the page}
  \justexplain{b}{bottom of the page}
  \justexplain{p}{on a special page}
\end{longtable}

Two figures inline:
\begin{latex}
\begin{figure}[!hbt]
  \centering
  \includegraphics[height = 3cm]{stema.eps}
  \hspace{3cm}
  \includegraphics[width = 3cm]{Escher_Relativity.jpg}
  \caption[Imagini\^{i}n linie]{Stema Facult\u{a}\c{t}ii de Matematic\u{a}
    \c{s}i \textit{Relativitate} de M.C. Escher (1953)}\label{Stema_Escher2}
\end{figure}
\end{latex}

\note{We can use minipages to number them separately.}

To number them separately but within the same main number use \code{subfigure} 
(Requires the package \code{subfigure}):
\begin{latex}
\begin{figure}[!htb]
  \centering
  \subfigure[Stema Facult\u{a}\c{t}ii de Matematic\u{a}]{
    \includegraphics[height=3cm]{stema.eps}\label{Stema4}}
  \hspace{3cm}
  \subfigure[\textit{Relativitate} de M.C. Escher (1953)]{
    \includegraphics[width=3cm]{Escher_Relativity.jpg}\label{Escher4}
  }\\
  \caption{Stema Facult\u{a}\c{t}ii de Matematic\u{a}
    \c{s}i \textit{Relativitate} de M.C. Escher (1953)}\label{Stema_Escher4}}
\end{figure}
\end{latex}
\subsection{Fancy quotes \& Unicode}
\begin{latex}
\usepackage[T1]{fontenc}
% allows inserting unicode directly in latex (i.e. from keyboard)
\usepackage[utf8]{inputenc} % \noncurs
\end{latex}

\begin{longtable}{l c l}
\explain{\quotesinglbase}{Single low-9 quotation mark}
\explain{\quotedblbase}{Double low-9 quotation mark}
\explain{\guillemetleft}{Left-pointing double angle quotation mark}
\explain{\guillemetright}{Right-pointing double angle quotation mark}
\explain{\guilsinglleft}{Single left-pointing angle quotation mark}
\explain{\guilsinglright}{Single right-pointing angle quotation mark}
\explain{'}{Single high quotation mark}
\explain{`}{Single high-6 quotation mark}
\explain{``}{Double high-6 quotation mark}
\explain{''}{Double high-9 quotation mark}
\explain{"}{Double high-straight quotation mark}
\end{longtable}

\begin{example}
\begin{center}
\quotedblbase Cze\'s\'c'' czy ,,dzien dobry''. \\
\textquotedblleft Hello\textquotedblright{} 
or ``Hi''.\\
\guillemetleft Bonjour\guillemetright{}.
\end{center}
\end{example}

\subsection{Centering and setting font sizes}
Quick Usage:
\begin{latex}
\usepackage{sectsty}
\chaptertitlefont{\centering\LARGE}
\sectionfont{\Large}
\end{latex}

Long list (tl;dr - executes ... before printing whatever it says): \noncurs
\vspace{-11pt}
\begin{multicols}{2} \noindent
  \code{\allsectionsfont{...}} \\
  \code{\partfont{... }}\\
  \code{\chapterfont{...}}\\
  \code{\sectionfont{...}}\\
  \code{\subsectionfont{...}}\\
  \code{\subsubsectionfont{...}}\\
\columnbreak
  \code{\paragraphfont{...}}\\
  \code{\subparagraphfont{...}}\\
  \code{\minisecfont{...}}\\
  \code{\partnumberfont{...}}\\
  \code{\parttitlefont{...}}\\
  \code{\chapternumberfont{...}}\\
  \code{\chaptertitlefont{...}}
\end{multicols}

\subsection{Links and references}
\begin{latex}
\usepackage{hyperref}
\hypersetup{colorlinks=true, linkcolor=cyan, citecolor=green,
  filecolor=black, urlcolor=blue}
\end{latex}
More in \nameref{labels} (page \pageref{labels})

\subsection{Dots after chapters in table of contents}
\package{tocloft}\\
\code{\renewcommand{\cftchapleader}{\cftdotfill{\cftdotsep}}}

If we are in \article, we use:\\
\code{\renewcommand{\cftsecleader}{\cftdotfill{\cftdotsep}}}

\note{if this command is misspelled the error doesn't appear immediately,
but when a chapter is added to the Table of Contents.}

