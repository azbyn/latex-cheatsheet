\vspace{-1cm}
\section{References and labels}\label{labels}
Requires \package{hyperref}.

\subsection{Labels}
\codeshow{\label{aLabel}}\\
\codeshow{\ref{aLabel}}\\
\codeshow{\ref*{aLabel}}\\
\codeshow{\nameref{aLabel}}\\
\codeshow{\nameref*{aLabel}}\\
\codeshow{\pageref{aLabel}}\\
\codeshow{\autoref{aLabel}}\\
\codeshow{\hyperref[aLabel]{here}}\\
% \codeshow{\hyperref{aLabel}}
\codeshow{\url{run:/usr/bin/firefox}}\\
\codeshow{\href{run:/usr/bin/firefox}{here}}\\
\codeshow{\cite{author/19}}\\
\codeshow{\cite[Chapter IV]{author/19}}\\
\codeshow{\eqref{aLabel}}

Adding "*" to \code{\...ref} will not create a clickable thing (the text is black).

Not adding \code{\phantomsection} before a link or reference might not center the page properly\\
e.g. \code{\phantomsection\label{bLabel}}


\subsection{Web Links}
\codeshow{\href{http://example.com/}{here}}\\
\codeshow{\url{http://example.com/}}\\
\codeshow{\href{mailto:my_addr@a.com}{my\_addr@a.com}}\\
\code{\nolinkurl} makes it so \LaTeX{} doesn't complain about an invalid url\\
\codeshow{\href{mailto:my_addr@a.org}{\nolinkurl{my_addr@a.org}}}

\subsection{Settings}
Using \code{\usepackage[hidelinks]{hyperref}} will make links black.
\subsubsection{Short way}
\vspace{-.5cm}
\begin{latex}
\hypersetup{colorlinks=true, linkcolor=cyan, citecolor=green,
  filecolor=black, urlcolor=blue}
\end{latex}
\subsubsectionNoncurs{Full way}
\begin{latex}
\hypersetup{
    bookmarks=true,         % show bookmarks bar?
    unicode=false,          % non-Latin characters in Acrobat’s bookmarks
    pdftoolbar=true,        % show Acrobat’s toolbar?
    pdfmenubar=true,        % show Acrobat’s menu?
    pdffitwindow=false,     % window fit to page when opened
    pdfstartview={FitH},    % fits the width of the page to the window
    pdftitle={My title},    % title
    pdfauthor={Author},     % author
    pdfsubject={Subject},   % subject of the document
    pdfcreator={Creator},   % creator of the document
    pdfproducer={Producer}, % producer of the document
    pdfkeywords={keyword1, key2, key3}, % list of keywords
    pdfnewwindow=true,      % links in new PDF window
    colorlinks=false,       % false: boxed links; true: colored links
    linkcolor=red,          % color of internal links
    citecolor=green,        % color of links to bibliography
    filecolor=magenta,      % color of file links
    urlcolor=cyan,          % color of external links
    %if colorlinks=false:
    linkbordercolor={1 0 0},% color of frame around internal links
    citebordercolor={0 1 0},% color of frame around citations
    urlbordercolor={0 1 1}  % color of frame around URL links
}  
\end{latex}
\note{The explicit RGB specification is only allowed for the border colors (like linkbordercolor etc.), while the others may only assigned to named colors.}