\section{Tables}
\subsection{The \texttt{tabular} environment}
\begin{latex}
\begin{tabular}[pos]{table spec}
\end{latex}
  
\note{a very common mistake is not defining the centering (ie \code{\begin{tabular}{l l}\end{tabular} } ) }

The \coden{table spec} argument tells LaTeX the alignment to be used in each column and the vertical lines to insert.

The number of columns isn't specified, since it's inferred by looking at the number of arguments provided.

Things you can add inside \coden{table spec}:

\begin{longtable}{l l}
  \justexplain{l}{left justified column}
  \justexplain{c}{center justified column}
  \justexplain{r}{right justified column}
  \justexplain{p{'width'}}{paragraph column with text vertically aligned at the top }
  \justexplain{m{'width'}}{paragraph column with text vertically aligned in the middle (requires array package) }
  \justexplain{b{'width'}}{paragraph column with text vertically aligned at the bottom (requires array package) }
  \justexplain{|}{vertical line}
  \justexplain{||}{double vertical line}
\end{longtable}

To specify a font format (such as bold, italic, etc.) for an entire column,
you can add \code{>{\format}} before you declare the alignment.
For example \code{ \begin{tabular}{ >{\bfseries}l c >{\itshape}r }}.
This requires the package \coden{array}.

\begin{example}
\begin{tabular}{|l|c|r || p{1cm}|m{1cm}|b{1cm}|}
l & c & r & p & m &b\\
\hfill l \hfill & \hfill c &\hfil r \hfil &
\hfill p \hfill & \hfill m &\hfil b \hfil\\
aaa & bbb & ccc &
AAA BBB CCC DDD&
AAA BBB CCC DDD&
AAA BBB CCC DDD\\
\end{tabular}
\end{example}

By default, if the text in a column is too wide for the page, LaTeX won’t automatically wrap it.
Using \coden{p{'width'}} you can define a special type of column which will wrap-around the text as in a normal paragraph.

In \coden{p} \coden{m} \code{b} the text is always left aligned. we could use \code{\hfill} to override this,
sometimes \code{\hfil} can also be useful.

Multiple columns can be defined at the same time using \code{*{count}{table spec}}.
E.g:\\
\begin{example}
\begin{tabular}{*{2}{l} *{2}{|c} |}
a & b & c& d\\
aaa & bbb & ccc& ddd
\end{tabular}
\end{example}

The optional parameter \code{pos} specifies the vertical position relative to the baseline of the surrounding text.
You can use:\\
\begin{tabular}{l l}
\justexplain{b}{bottom}
\justexplain{c}{center (default)}
\justexplain{t}{top}
\end{tabular}

For adding horizontal lines we use \code{\hline} or \code{\hline\hline} for a double line.
                     
\begin{example5}
Foo \quad
\begin{tabular}[b]{|l|| r | c|}\hline
Name & Math & LaTeX \\[0.3cm]\hline\hline
\L{}ucasz & 8 & 9 \\\hline
Wojtek & 12 & 1 \\\hline
\end{tabular} bar \ldots
\end{example5}

\begin{examplef}
Foo \quad
\begin{tabular}[c]{|m{2cm}|| m{1.5cm} | m{1.5cm}|}\hline
Country & Flammen\-werfer & Panzerkampfwagen  \\\hline\hline
Russia & - & - \\\hline
Germany & \checkmark & \checkmark \\\hline
\end{tabular} bar \ldots
\end{examplef}

\subsection{\texttt{wraptable}}\label{wraptable}
This requires the package \coden{wrapfig}
Usage:
\begin{latex}
\begin{wraptable}[number_of_lines]{pos}{width}    
\end{latex}

\begin{itemize}
\item \code{number_of_lines} The number of lines. May be omitted.
\item \code{pos} The horizontal alignment; you can use \code{l} or \code{r}.
\item \code{width} Width to be reserved for the table.
\end{itemize}

\begin{examplef}
We have a nice table below:

\begin{wraptable}[9]{r}{12cm}
\vspace{-11pt}
\hspace{20pt}
\begin{tabular}[c]{|l|| *9{l|} }\hline
  Country & Best song & Known for \\\hline\hline
  Russia & Siuda Kalatuszek & Putin's lifetime rule \\\hline
  Serbia & Crni Bombarder & Rakija and Kosovo \\\hline
  Poland & Hej soko\l{}y & Bisons and their grass \\\hline
  Finland & S\"akkij\"arven polkka & Suomi Perkele \\\hline
  Germany & Westerwaldlied & Losing both world wars \\\hline
\end{tabular}
\hspace{5pt}
\end{wraptable}
I have no idea what to put here, so I'll put this:
Nuapurista kuulu se polokan tahti jalakani pohjii kutkutti.
Ievan \"aiti se tytt\"o\"os\"a vahti vaan kyll\"ah\"an Ieva sen jutkutti,
sill\"a ei meit\"a silloin kiellot haittaa kun my\"o tanssimme laiasta laitaan.
Salivili hipput tupput t\"appyt \"appyt tipput hilijalleen.
\end{examplef}

\note{We can't use \code{wraptable} inside theorem environments}

\note{We can use \code{wrapfigure} with a similar syntax for inserting figures}

\code{\cline{i-j}} draws a line from column \coden{i} to \coden{j}.

\subsection{\code{multirow}s and \code{multicolumn}s}
This requires the package \coden{multirow}.
Usage:\
\begin{latex}
  \multicolumn{count}{alignment}{content}
  \multirow{count}{width}{content}
\end{latex}

Where:
\begin{itemize}
\item \code{alignment} can be \coden{l}, \coden{c}, \coden{r}.
\item \code{width} is the minimum cell width, can be \code{*} to be adjusted automatically.
\end{itemize}

\begin{examplef}
\begin{center}
\begin{tabular}{| l | *{3}{l|}l |} \hline
  \multirow{2}{*}{Country} & \multicolumn{3}{c|}{Food} & \multirow{2}{*}{Beverage}\\
  \cline{2-4}              & Soup & Main dish & Dessert & \\ \hline\hline
  Germany & Biersuppe & Bratwurst & Lebkuchen & Weizenbier\\\hline
  Russia  & Borscht & Stroganoff & Syrniki & Kvass\\\hline
  Poland  & Ch\l{}odnik & Go\l{}\k{a}bki & P\k{a}czek & Gorza\l{}ka \\\hline
\end{tabular}
\end{center}
\end{examplef}

If we want to set the min width by writing \code{\multirow{2}{3cm}{Country}} we need to use
\code{\hfill} (or \code{\hfil}) to center it. \code{\centering} can also be used

\begin{example}
\begin{tabular}{| l | l|} \hline
  \multirow{2}{3cm}{\centering Country} & A \\
  \cline{2-2}&              B \\ \hline\hline
  
  \multirow{2}{3cm}{\hfill Country \hfill} & A \\
  \cline{2-2}&              B \\ \hline\hline
  
  \multirow{2}{3cm}{\hfil Country \hfil} & A \\
  \cline{2-2}&              B \\ \hline\hline
  
  \multirow{2}{3cm}{Country} & A \\
  \cline{2-2}&              B \\ \hline
\end{tabular}
\end{example}

\subsection{\texttt{arraystrectch}}

Redefining spacing between lines \code{\renewcommand{\arraystretch}{factor}}

\begin{example}
\renewcommand{\arraystretch}{1.5}
\begin{tabular}{| l | l|} \hline
  \multirow{2}{3cm}{\centering Country} & A \\
  \cline{2-2}&              B \\ \hline
\end{tabular}
\end{example}

If we want to scale the table: requires \code{graphics}\\
\begin{example}
  \resizebox{7cm}{1.4cm}{
  \renewcommand{\arraystretch}{1.5}
  \begin{tabular}{|l|l||c|c||*{3}{c|}}\hline
    \multicolumn{2}{|c||}{ }&
    \multicolumn{2}{c||}{Regimul disciplinei}&
    \multicolumn{3}{c|}{Numar de ore/sapt.}\\
    \cline{3-7} \multicolumn{2}{|c||}{ } &
    Obligatorie& Optionala & Curs &
    Sem.  & Lab.\\ \hline\hline
    \multirow{3}{*}{Sem.  I} & Disciplina 1 &
    \checkmark& -- & 3 & 2 & 1\\ \cline{2-7}&
    Disciplina 2 &\checkmark & -- & 2 &
    1 & 1\\\cline{2-7}
    & Disciplina 3 & -- &\checkmark &
    2 & 2 & -\\\cline{1-7}
    \multirow{2}{*}{Sem.  II} &
    Disciplina 4 &\checkmark&
    -- & 2 & - & 3\\ \cline{2-7}
    & Disciplina 5 & -- &\checkmark &
    2 & 2 & -\\ \hline
  \end{tabular}
\renewcommand{\arraystretch}{1}
}
\end{example}

\begin{examplef}
\begin{tabular}{cc|c|c|c|c|l}
\cline{3-6}
& & \multicolumn{4}{ c| }{Primes} \\ \cline{3-6}
& & 2 & 3 & 5 & 7 \\ \cline{1-6}
\multicolumn{1}{ |c  }{\multirow{2}{*}{Powers} } &
\multicolumn{1}{ |c| }{504} & 3 & 2 & 0 & 1 &     \\ \cline{2-6}
\multicolumn{1}{ |c  }{}                        &
\multicolumn{1}{ |c| }{540} & 2 & 3 & 1 & 0 &     \\ \cline{1-6}
\multicolumn{1}{ |c  }{\multirow{2}{*}{Powers} } &
\multicolumn{1}{ |c| }{gcd} & 2 & 2 & 0 & 0 & min \\ \cline{2-6}
\multicolumn{1}{ |c  }{}                        &
\multicolumn{1}{ |c| }{lcm} & 3 & 3 & 1 & 1 & max \\ \cline{1-6}
\end{tabular}
\end{examplef}

\subsection{Colored tables}
This requires the package \coden{colortbl}.

This defines: \code{\rowcolor{red}}, \code{\cellcolor{blue}} and
\code{\columncolor{MidnightBlue}}

\begin{example}
\begin{tabular}{l | l | >{\columncolor{red}} l | l}
\rowcolor{red} A & B  &\cellcolor{blue} C & D\\ 
%A & \rowcolor{blue} A & C \\ 
A & \cellcolor{green} B & C & D\\ 
A & B & C & D\\ 
\end{tabular}
\end{example}

\subsection{\texttt{table} environment}
This is very similar to \texttt{figure} (page \pageref{figure}).
\begin{latex}
\begin{table}[pos]
\end{latex}

Where \coden{pos} can be \coden{h} - here, \coden{t} - top, \coden{b} - bottom.
It's useful to set \coden{pos} to \coden{!hbt} meaning force it here, if not possible put it on the bottom,
else at the top.

\begin{examplefr}
\begin{centering}
\begin{table}[h]
  \caption{A nice table}\label{nice_table}
  \begin{tabular}{l | l}
    this & is\\
    vert & nice\\
  \end{tabular}
\end{table}
\end{centering}

This is a nice table: \ref{nice_table}.
\end{examplefr}

\note{\coden{\listoftable} will print a list of tables}

\subsection{\texttt{tabularx}}
This requires the package \coden{tabularx}.

It stretches the table to 
\begin{examplef}
\begin{tabularx}{\textwidth}{ |X|X|X|X| }
  \hline
  label 1 & label 2 & label 3 & label 4 \\
  \hline 
  item 1  & item 2  & item 3  & item 4  \\
  \hline
\end{tabularx}
\end{examplef}

\subsection{\texttt{longtable}}
This requires the package \coden{longtable}.

\begin{latex}
\begin{longtable}[pos]{table spec}
\end{latex}

Does everything that \coden{tabular} does and more:
\begin{itemize}
\item can be split on multiple pages
\item can be labeled
\item is centered (by default), by setting the optional argument \coden{pos} to \coden{r} or \coden{l} we can change this
\item can have captions
\end{itemize}

\begin{example}
\begin{longtable}{l l}\label{longTable}
  We finally & finished tables \\
  yay & yay \\
\end{longtable}
\bigskip
Table \ref{longTable} is a nicer table.
\end{example}

\note{\texttt{sidewaystable} might be useful for sideways tables}.
