\section{Symbols}
\subsection*{Regular}
\codeshow{$+ - = ! / ( ) [ ] < > | ' : *$}
\subsection*{Greek}
Only showing peculiar symbols. Most have intuitive names.\\
\begin{example}
\[\begin{matrix}
  \gamma & \eta    & \kappa\\
  \mu    & \nu     & \varphi\\
  \rho   & \varrho & \chi
\end{matrix}\]
\end{example}
\subsection*{Operators}
Most multiple letter functions have a separate command (eg. \mcodeshow{\sin}).

Other peculiar symbols are:\\
\mcodeshow{\liminf}\\
\mcodeshow{\limsup}

For declaring custom operators:
\mcodeshow{\operatorname{atan}(x)}\\
or if it's used frequently: \code{\DeclareMathOperator*{\atan}{atan}} (in preamble).

\subsection*{Sums, integrals, and other ``big'' symbols}

\begin{longtable}{l l | l l | l l}
  \codeshowB{\sum      }{\prod      }{\coprod}
  \codeshowB{\bigoplus }{\bigotimes }{\bigodot}
  \codeshowB{\bigcup   }{\bigcap    }{\biguplus}
  \codeshowB{\bigsqcup }{\bigvee    }{\bigwedge}
  \codeshowB{\int      }{\oint      }{\iint}
  \codeshowB{\iiint    }{\iiiint    }{\idotsint}
\end{longtable}

\subsection*{Fancy braces}
\begin{example}
\[
( a ), [ b ], \{ c \}, | d |, \| e \|,
\langle f \rangle, \lfloor g \rfloor,
\lceil h \rceil, \ulcorner i \urcorner
\]
\end{example}

\subsection*{Fractions}

\begin{example}
\[
x = a_0 + \cfrac{1}{a_1 
          + \cfrac{1}{a_2 
            + \cfrac{1}{a_3 +\cfrac{1}{a_4} } } }
\]
\end{example}

vs
    
\begin{example}
\[
x = a_0 + \frac{1}{a_1 
          + \frac{1}{a_2 
            + \frac{1}{a_3 + \frac{1}{a_4} } } }
\]
\end{example}

There's also the option for slanted fractions (requires \package{xfrac}):\\\codeshow{\sfrac{1}{2}}

\subsection*{Automatic sizing}
\begin{example}
\[
P\left(A=2\middle|\frac{A^2}{B}>4\right)
\]
\end{example}

\subsection*{Manual sizing}
\begin{example}
\[
  ( \big( \Big( \bigg( \Bigg(
  \frac{\mathrm d}{\mathrm d x} \big( k g(x) \big)
\]
\end{example}

\subsection*{Accents}
\begin{longtable}{l l | l l}
  \code{a' or a ^{}} & $a'$ & \mtshow{a''}
  \codeshowD{\hat{a}}{\bar{a}}\\
  \codeshowD{\grave{a}}{\acute{a}}\\
  \codeshowD{\dot{a}}{\ddot{a}}\\
  \codeshowD{\overrightarrow{AB}}{\overleftarrow{AB}}\\
  \codeshowD{\overline{aaaa}}{\check{a}}\\
  \codeshowD{\breve{a}}{\vec{a}}\\
  \codeshowD{\dddot{a}}{\ddddot{a}}\\
  \codeshowD{\widehat{ABC}}{\widetilde{AAA}}\\
  \codeshowD{\tilde{a}}{\underline{a}}\\
  \codeshowD{\underset{u}{abc}}{\overset{o}{abc}}\\
  \codeshowD{\underbrace{abc}}{\overbrace{abc}}\\
  \mtshow{\stackrel\frown{AAA}}
\end{longtable}
E.g.:\\
\begin{example}
\[
  \overline { \overline{A} \cup
    \overline{\overline{B}} }
  = A \cap \overline{B} \, ,\quad
  \text{folosind legile lui De Morgan}
\]
\end{example}

\subsection*{Dots}
\begin{longtable}{l l l}
  \explain{\dots}{generic dots (ellipsis), to be used in text (outside formulae as well).
    It automatically manages whitespaces before and after itself according to the context. }
  \explainM{\ldots}{similar to the previous, but no whitespace management}
  \explainM{\cdots}{centered dots}
  \explainM{\vdots}{vertical dots}
  \explainM{\ddots}{diagonal dots}
  \explainM{\iddots}{inverse diagonal dots (requires the package \coden{mathdots})}
  \coden{\hdotsfor{n}}& $\dots\dots$ & a row of dots spanning $n$ columns.
\end{longtable}

\subsection*{Other operators}
\subsubsection*{Binary and relation}
\begin{longtable}{l l | l l | ll | ll}
  \codeshowD{<}{>} & \codeshowD{=}{\doteq}\\
  \codeshowD{\leq}{\geq} & \codeshowD{\equiv}{\approx}\\
  \codeshowD{\ll}{\gg} & \codeshowD{\cong}{\simeq}\\
  \codeshowD{\subset}{\supset} & \codeshowD{\sim}{\propto}\\
  \codeshowD{\subseteq}{\supseteq} & \codeshowD{\parallel}{\nparallel}\\
  \codeshowD{\nsubseteq}{\nsupseteq} & \codeshowD{\asymp}{\bowtie}\\
  \codeshowD{\sqsubset}{\sqsupset} & \codeshowD{\vdash}{\dashv}\\
  \codeshowD{\sqsubseteq}{\sqsupseteq} & \codeshowD{\smile}{\frown}\\
  \codeshowD{\preceq}{\succeq} & \codeshowD{\prec}{\succ}\\
  \codeshowD{\in}{\ni} & \codeshowD{\notin}{\neq}\\
  \codeshowD{\mid}{\perp} & \codeshowD{\models}{\therefore}\\
  \codeshowD{\sphericalangle}{\measuredangle} & \codeshowD{\leqslant}{\geqslant}\\
  \codeshowD{\subsetneq}{\varsubsetneq} & \codeshowD{\subsetneqq}{\subseteqq}\\
  \codeshowD{\nleq}{\ngeq} & \codeshowD{\nsim}{\nleqslant}\\
\end{longtable}
\begin{longtable}{l l | l l | ll | ll}
  \codeshowD{\pm}{\mp}& \codeshowD{\cup}{\cap}\\
  \codeshowD{\div}{\times}&\codeshowD{\vee}{\wedge}\\
  \codeshowD{\ast}{\star}&\codeshowD{\sqcup}{\sqcap}\\
  \codeshowD{\setminus}{\smallsetminus}&\codeshowD{\uplus}{\wr}\\
  
  \codeshowD{\dagger}{\ddagger}&\codeshowD{\bigtriangleup}{\bigtriangledown}\\
  \codeshowD{\cdot}{\odot}&\codeshowD{\triangleleft}{\diamond}\\
  \codeshowD{\oplus}{\ominus}&\codeshowD{\circ}{\amalg}\\
  \codeshowD{\oslash}{\otimes}&\codeshowD{\bigcirc}{\bullet}
\end{longtable}
\subsubsection*{Logic and arrows}
\hspace{-1cm}
\begin{tabular}{l l | l l | ll | ll}
  \codeshowD{\exists}{\nexists}&
  \coden{\rightarrow} or \coden{\to} & $\to$ &
  \coden{\leftarrow} or \coden{\gets} & $\gets$ \\
  \codeshowD{\forall}{\neg}& \codeshowD{\mapsto}{\leftrightarrow}\\
  \codeshowD{\land}{\lor}& \codeshowD{\leftrightharpoons}{\leftrightarrows}\\
  \codeshowD{\top}{\bot}&\codeshowD{\implies}{\Rightarrow}\\
  \codeshowD{\emptyset}{\varnothing}&\codeshowD{\iff}{\Leftrightarrow}\\
  \codeshowD{\langle}{\rangle}& \codeshowD{\vert}{\Vert}\\
  \codeshowD{\angle}{\backslash}&\codeshowD{\mid}{\|}\\
  \codeshowD{\uparrow}{\downarrow} & \codeshowD{\Uparrow}{\Downarrow} \\
  
  \codeshowD{\nrightarrow}{\longmapsto} & \codeshowD{\varsubsetneq}{\Leftarrow}\\
  \codeshowD{\leadsto}{\updownarrow} & \codeshowD{\longrightarrow}{\Longrightarrow}\\
  \codeshowD{\nearrow}{\searrow} & \codeshowD{\swarrow}{\nwarrow}\\
\end{tabular}
\subsubsection*{Other symbols}
\begin{longtable}{l l | l l | ll | ll | ll | ll}
  \codeshowD{\partial}{\imath}&
  \codeshowD{\Re}{\nabla}&
  \codeshowD{\aleph}{\square}\\
  
  \codeshowD{\eth}{\jmath}&
  \codeshowD{\Im}{\Box}&
  \codeshowD{\beth}{\blacksquare}\\

  \codeshowD{\hbar}{\ell}&
  \codeshowD{\wp}{\infty}&
  \codeshowD{\gimel}{\&}\\
  
  \codeshowD{\sharp}{\checkmark}&
  \codeshowD{W}{W}&
  \codeshowD{W}{WIP}
\end{longtable}

\subsubsection*{Misc}

\begin{longtable}{l l}
\mtshow{x \equiv a \pmod{b}}
\mtshow{a \bmod{b}}
\mtshow{\sqrt[5]{n}}
\mtshow{\frac{a}{b}}
\mtshow{\left( \frac{a}{b} \right)}
\mtshow{\left\{ \frac{a}{b} \right.}
\end{longtable}

% add def eq