\section{Math mode}
\begin{itemize}
\item \textbf{Inline} can be entered by using \code{$...$} or \code{\(...\)}, or by using
  the environment \code{math}
\item \textbf{Displayed} mode can be entered using \code{\[...\]} or with the environment
  \code{displaymath}.\\
  Using the environment \code{equation} automatically numbers the equation.\\
  \code{equation*} and \code{displaymath} are functionally equivalent
\end{itemize}
\codeeg{\[F(x) = \int f(x)\, dx \]}\\
\codeeg{$F(x) = \int f(x)\,dx$}

\note{Adding \code{fleq} in the \code{\documentclass} options aligns equations to the left}

Most spaces are ignored, and must be specified manually.\\
\note{LaTeX complains if you leave empty lines in math mode.}

Adding \code{\numberwithin{equation}{section}} will reset equation numbers every section.
\code{chapter} may also be used.

\subsection{Bold Math}
\begin{example}
  this doesn't work: \textbf{Text $f$}\\
  this works: \textbf{Text} $\boldsymbol{f}$
\end{example}
\subsection{Font size}
\begin{longtable}{l l }
  \code{\displaystyle} & $\displaystyle \int$\\
  \code{\textstyle} & $\textstyle \int$\\
  \code{\scriptstyle} & $\scriptstyle \int$\\
  \code{\scriptscriptsyle} & $\scriptscriptstyle  \int$
\end{longtable}

Declaring \code{\everymath{\displaystyle}} in the preamble will force \code{\displaystyle}
in all math environments


