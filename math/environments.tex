\section{Environments}
\note{Adding \code{\nonumber} to an otherwise numbered environments cancels the numbering.}\\
\note{Adding more white space than needed inside \code{align} and \code{aligned} makes latex complain.}

Since ``a picture is worth a thousand words'', just look at this:

\subsection{align}
Writing \code{\allowdisplaybreaks} in the preamble allows splitting an equation written in \code{align} on multiple pages.

\begin{example}
\begin{align}
  a & = 3\\
    & = 2 + 1 \nonumber \\
    & = 2 + \sigma(0)
\end{align}
\end{example}

Adding small interjections to align:\\
\begin{example}
\begin{minipage}{3in}
\begin{align*}
\intertext{If}
   A &= \sigma_1+\sigma_2\\
   B &= \rho_1+\rho_2\\
\intertext{then}
C(x) &= e^{Ax^2+\pi}+B
\end{align*} 
\end{minipage}
\end{example}

\subsection{aligned}
Like align but used inside math mode. Useful for numbering multiple lines once.

\begin{example}
\begin{equation}
\begin{aligned}
  \pi & = 3\\
    & = 2 + \sigma(0)
\end{aligned}
\end{equation}
\end{example}

\subsection{eqnarray}
People on the internet say ``don't use \code{eqnarray}'' so we won't talk about it.

\subsection{subequations}
\begin{example}
\begin{subequations}
\begin{align}
  \pi &= 3\\
  &= 2 + \sigma(0)
\end{align}
\end{subequations}
\end{example}

or:

\begin{example}
\begin{subequations}
\begin{equation}
  \pi = 3\label{blasphemy}
\end{equation}
Therefore:
\begin{equation}
  \int_0^1\sqrt{1-x^2} dx = \frac{3}{4}
\end{equation}
\end{subequations}

The equation \eqref{blasphemy} %(\ref{blasphemy})
is obviously false\footnote{
  Unless you are an engineer}.
\end{example}


\subsection{cases}
Useful for functions with multiple branches.

\begin{example}
\[f(x) =
\begin{cases}
  \int\frac{\sin(x)}{x},&\text{daca } x \neq 0,\\
  1,&\text{daca } x=0.
\end{cases}
\]
\end{example}

With displaystyle:\\
\begin{example}
\[f(x) =
\begin{dcases}
  \int\frac{\sin(x)}{x},&\text{daca } x \neq 0,\\
  1,&\text{daca } x=0.
\end{dcases}
\]
\end{example}

If there's a lot of plain text on the right\footnote{same goes for \code{cases*}}:\\
\begin{example}
\[f(x) =
  \begin{dcases*}
    \int \frac{\sin(x)}{x},&daca $x \neq 0$,\\
    1,& daca $x=0$.
  \end{dcases*}
\]
\end{example}


\newcommand{\parti}[2]{\frac{\partial #1}{\partial #2}}
Nice spacing:\\
\begin{example}
%\newcommand{\parti}[2]{\frac{\partial #1}
%  {\partial #2}}
\[f(x, y) =
\begin{dcases}
\parti{g}{x}\,,&\text{daca } x y \geq 0,\\[.5em]
\parti{g}{y}\,,&\text{daca } x y < 0,
\end{dcases}
\]
\end{example}

vs\\
\begin{example}
\[f(x, y) =
\begin{dcases}
 \parti{g}{x},&\text{daca } x y \geq 0,\\
 \parti{g}{y},&\text{daca } x y < 0,
\end{dcases}
\]
\end{example}



\subsection{matrix}
For writing matrices with more columns (default is 10):
\code{\setcounter{MaxMatrixCols}{15}}

\begin{tabular}{|l|l|}
  \hline
  Environment name & Surrounding delimiter\\
  \hline
pmatrix & $( )$\\
bmatrix & $[ ]$\\
Bmatrix & $\{ \}$\\
vmatrix & $| |$\\
Vmatrix & $\Vert \Vert$\\
  \hline
\end{tabular}

\begin{example}
\[
A_{m,n} = 
 \begin{pmatrix}
  a_{1,1} & a_{1,2} & \cdots & a_{1,n} \\
  a_{2,1} & a_{2,2} & \cdots & a_{2,n} \\
  \vdots  & \vdots  & \ddots & \vdots  \\
  a_{m,1} & a_{m,2} & \cdots & a_{m,n} 
\end{pmatrix}
\]
\end{example}

Nicer spacing:\\
\begin{example}
\begin{equation*}
  \begin{bmatrix*}[r]
    x    & -y+y' & z+z'\medskip  \\
    u    &  v    & w   \medskip  \\
    -r-r'&  s    & -t
  \end{bmatrix*}
\end{equation*}
\end{example}

\code{hline} adds a horizontal line. Adding a \code{|} like below adds a vertical line.\\
\begin{example}
\[
\begin{array}{c|c}
  1 & 2 \\ 
  \hline
  3 & 4
 \end{array}
\]
\end{example}

\begin{example}
\[
M = \bordermatrix{~ & x & y \cr
                  A & 1 & 0 \cr
                  B & 0 & 1 \cr}
\]
\end{example}

\begin{example}
A matrix in text must be set smaller:
$\bigl(\begin{smallmatrix}
a&b \\ c&d
\end{smallmatrix} \bigr)$
to not increase leading in a portion of text.
\end{example}


\begin{example}
\[
  \boldsymbol{\beta} =
     (\beta_1,\beta_2,\dotsc,\beta_n)
\]
\end{example}

Adding a "*" after the name allows us to specify alignment:

\begin{example}
\[
\begin{matrix}
  -1 & 3 \\
  2 & -4
 \end{matrix}
 =
 \begin{matrix*}[r]
  -1 & 3 \\
  2 & -4
 \end{matrix*}
\]
\end{example}