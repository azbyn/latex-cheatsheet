\section{Theorem environments}
\note{\code{\emph} might be useful inside theorems}
\subsection{Without \code{asmthm}}
\begin{latex}
\newtheorem{theo}{Teorema}
%the [theo] signifies that all these environments share the same counter
\newtheorem{corol}[theo]{Corolarul}
\newtheorem{defin}[theo]{Defini\c{t}ia}
\newtheorem{exem}[theo]{Exemplul}
\newtheorem{exer}[theo]{Exerci\c{t}iul}
\newtheorem{lema}[theo]{Lema}
\newtheorem{prop}[theo]{Propozi\c{t}ia}
\newtheorem{rem}[theo]{Remarca}
\newenvironment{proof}{\noindnt\textbf{Demonstratie.}}{\hfill\rule{.5em}{.5em}}
\end{latex}

That square is made using \code{\rule[raise]{width}{thickness}} (raise means
above or below the baseline)


This will number theorems continuously throughout the document.
To reset the counter each chapter/ section define like so:
\code{\newtheorem{theo}{Teorema}[chapter]}

Adding \code{[Some name]} will name that theorem.

\begin{example}
\begin{theo}[Trichotomy theorem]
  Let $x \in \R, y \in \R$, $(x = y) \lor (x < 0)
  \lor (x > 0)$.
\end{theo}
\begin{proof}
  The proof is trivial and left as
  an exercise to the reader.
\end{proof}
\begin{corol}
$x \neq 0 \implies (x < 0) \lor (x > 0)$
\end{corol}
\end{example}

\subsection{With \code{asmthm}}
Adding a "*" after \code{\newtheorem} will not number any theorem of that type
like so:\\
\code{\newtheorem*{rem}[theo]{Remarca}}

Theorem styles:
\begin{longtable}{l l}
  Name & Appearance \\
  \hline
  \code{plain} & \textbf{Theorem 1.} \emph{Some text.} \\
  \code{definition} & \textbf{Definition 1.} Some text. \\
  \code{remark} & \textit{Remark 1.} Some text. \\
\end{longtable}
Custom theorems:\noncurs
\begin{latex}
\newtheoremstyle{stylename}% name of the style to be used
  {spaceabove}% measure of space to leave above the theorem. E.g.: 3pt
  {spacebelow}% measure of space to leave below the theorem. E.g.: 3pt
  {bodyfont}% name of font to use in the body of the theorem
  {indent}% measure of space to indent
  {headfont}% name of head font
  {headpunctuation}% punctuation between head and body
  {headspace}% space after theorem head; " " = normal interword space
  {headspec}% Manually specify head
\end{latex}

\begin{latex}
\theoremstyle{plain} %the default style
\newtheorem{theo}{Teorema}[section]
\newtheorem{corol}[theo]{Corolarul}
\newtheorem{prop}{Propozi\c{t}ia}[section]
\theoremstyle{definition}
\newtheorem{defin}{Defini\c{t}ia}[section]
\newtheorem{exem}{Exemplul}[section]
\end{latex}

Proof now becomes:
\begin{latex}
  \renewcommand*{\proofname}{\noindent\textbf{Demonstra\c{t}ie.}}
\end{latex}

To replace the Q.E.D. symbol:\noncurs
\begin{latex}
\renewcommand{\qedsymbol}{$\blacksquare$}
\end{latex}

\begin{example}
\begin{theo}
  This is a theorem $f = 0$.
\end{theo}
\begin{proof}
Here is the proof:
\[a^2 + b^2 = c^2 \qedhere\]
\end{proof}

\begin{proof}
Here is another proof:
\[a^2 + b^2 = c^2 \]
\end{proof}
\end{example}

Doing something like this might be useful to add a symbol at the end:
\begin{latex}
\newenvironment{exem}{\begin{example}}{\hfill$\diamond$\end{example}}
\end{latex}
