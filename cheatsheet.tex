\documentclass[11pt,a4paper]{report}

\usepackage{minted}

\usepackage{amsmath}
\usepackage{amssymb}
\usepackage{amsfonts}
\usepackage{mathrsfs}
\usepackage{mathtools}

\usepackage{fancyhdr}
\usepackage{makeidx}
 
%\usepackage{color}
\usepackage[dvipsnames]{xcolor}
\usepackage{graphicx}
\usepackage{epstopdf}
\usepackage{multicol}

\usepackage{hyperref}
\usepackage{geometry}

\usepackage{soul}
% \newcommand{\sout}[1]{#1}

\usepackage{combelow}

\usepackage[T1]{fontenc}
\usepackage[utf8]{inputenc}

\definecolor{bg}{gray}{0.96}
%\definecolor{bg}{HTML}{F0F0F0}
\newmintinline[code]{latex}{bgcolor=bg}
\newminted[latex]{latex}{bgcolor=bg}

\newcommand{\package}[1]{\code{\usepackage{#1}}\label{#1}}
\newcommand{\noncurs}{\hspace{.5cm}$\notin \mathscr{C}$}

\newcommand{\explain}[2]{\code{#1} & #1 & #2\\}
\newcommand{\justexplain}[2]{\code{#1} & #2\\}
\newcommand{\explainM}[2]{\code{#1} & $#1$ & #2\\}

\newcommand{\explainMFont}[2]{\explainM{#1 {ABC\ i\ abc\ 123}}{#2}}
\newcommand{\explainMFontUpper}[2]{\explainM{#1 {ABC\ I\ DEF}}{#2 (uppercase only)}}

\newcommand{\codeshow}[1]{\code{#1} - #1}
\newcommand{\codeeg}[1]{e.g. \codeshow{#1}}

\newcommand{\packageSubsection}[2]{\subsection[#1]{#1 - \code{#2}}}
\newcommand{\subsubsectionNoncurs}[1]{\subsubsection[#1]{#1 \noncurs}}
\newcommand{\subsectionNoncurs}[1]{\subsection[#1]{#1 \noncurs}}
\newcommand{\disclamer}[1]{\textbf{Disclamer:} #1}

%\usepackage[romanian]{babel}

\geometry{a4paper,left=25mm,right=20mm,top=20mm,bottom=30mm}

\newcommand{\book}{\code{book}\ }
\newcommand{\article}{\code{article}\ }
\newcommand{\report}{\code{report}\ }

\hypersetup{colorlinks=true, linkcolor=Blue, citecolor=green,
  filecolor=black, urlcolor=blue}
\usepackage{longtable} % To display tables on several pages

\renewcommand{\indexname}{I\lowercase{ndice}}

%\renewenvironment{longtable}[1]{\begin{longtable}[l]{#1}}{\end{longtable}}

\title{\LaTeX{} Cheatsheet}
\author{Łucasz Zieliński\thanks{Dzięki światu}}
\date{\today}

\makeindex
%%%%%%%%%%%%%%
\begin{document}

\begin{titlepage}
\maketitle
\end{titlepage}
\tableofcontents
%\newpage


\chapter{Das gute Zeug}
\section{Document classes \& differences}
\code{\documentclass[...]{report, article, book, beamer}}\\
e.g. \code{\documentclass[11pt,a4paper]{report}}

\subsection{Differences with regard to available commands and environments}
\begin{itemize}
\item \book and \report feature the \code{\chapter} sectioning command, while \article doesn't.
\item In \book and \report, \code{\appendix} will cause \code{\chapter}s to be typeset as "Appendix X" instead of "Chapter X". For article, this isn't applicable.
\item \book and \report will start a new page for \code{\part}s , while \article won't.
\item \book offers the \code{\frontmatter}, \code{\mainmatter}, and \code{\backmatter} commands to control page numbering (Roman for the front matter, arabic elsewhere) and numbering of sectioning titles (no numbering in the front and back matter), while \report and \article don't.
\item \book \textit{doesn't} offer the \code{abstract} environment, while \report and \article \textit{do}.
\end{itemize}
\subsection{Differences with regard to default settings}
\begin{itemize}
  \item The \book class uses the \code{twoside} class option (which means different margins and headers/footers for even and odd pages), while \report and \article use \code{oneside}.
  \item The \book class uses the \code{twoside} class option (which means different margins and headers/footers for even and odd pages), while \report and \article use \code{oneside}.
  \item \book uses \code{openright} (new parts and chapters start on "right" pages, adding a blank page before if necessary), while \report uses \code{openany}. (Note that "right" means an odd page in twoside mode, but any page in oneside mode.) For \article, the distinction between \code{openright} and \code{openany} isn't applicable.
    
  \item \book uses the \code{headings} pagestyle for non-chapter-starting pages, while \report and \article always use \code{plain}.
    
  \item \book and \report use \code{titlepage} (the title page and -- if applicable -- the \code{abstract} environment will be typeset on pages of their own), while \article uses \code{notitlepage}.
    
  \item For \book and \report, the lowest-level sectioning command which is numbered and incorporated into the table of contents is \code{\subsection}, while for article it is \code{\subsubsection}.
    
  \item \book and \report will use the arguments of \code{\chapter}s and \code{\section}s for running headings (if such headings are present), while \article will use \code{\section}s and \code{\subsection}s.
    
  \item \book and \report will number floats (figures, tables etc.), equations, and footnotes per chapter, while \article will number them continuously. Note that footnotes -- even when numbered per chapter -- do not feature a chapter prefix.

  \item \book and \report will use \code{\bibname} (which defaults to "Bibliography") for the heading of bibliographic references, while article will use \code{\refname} (which defaults to "References").
\end{itemize}

\section{Packages}
\begin{enumerate}
\item Math:\\
  \package{amsmath} \\
  \package{mathtools} \\
  \package{amssymb}\\
  \package{amsfonts}\\
  \package{mathrsfs} - for \code{\mathscr} \noncurs

\item Fancy page style:\\
  \package{fancyhdr}

\item Color:\\
  \package{color}\\
  or even better \code{\usepackage[dvipsnames]{xcolor}}

\item Images:\\
  \package{graphicx}\\
  \package{epstopdf}

\item Unicode - fancy quotes:\\
  \code{\usepackage[T1]{fontenc}}\\
  \code{\usepackage[utf8]{inputenc}} \noncurs

\item Index requires \texttt{Ctrl-Shift-I} in WinEdt: \\
  \package{makeidx}

\item Hyper references:\\
  \package{hyperref}

\item Colums: \noncurs\\
  \package{multicol}

\item Setting title formatting (font size, centering, etc.):\\
  \package{sectsty}
  
\end{enumerate}

\section{Package ussage}
\packageSubsection{Fancy Quotes}{\usepackage[T1]{fontenc}}
\code{,,} might work instead of \code{\quotedblbase}. \code{% ,,Hello''}.

\begin{longtable}{l c l}
\explain{\quotedblbase}{Double low-9 quotation mark}
\explain{\quotesinglbase}{Single low-9 quotation mark}
\explain{\guillemetleft}{Left-pointing double angle quotation mark}
\explain{\guillemetright}{Right-pointing double angle quotation mark}
\explain{\guilsinglleft}{Single left-pointing angle quotation mark}
\explain{\guilsinglright}{Single right-pointing angle quotation mark}
\end{longtable}

\codeeg{\quotedblbase Cześć''}

\codeeg{\textquotedblleft Hello\textquotedblright} or \codeshow{``Hi''}

\packageSubsection{Centering and seting font sizes}{\usepackage{sectsty}}
Quick Usage:\\
\code{\chaptertitlefont{\centering\LARGE}}\\
\code{\sectionfont{\Large}}
\\\\
Tl;dr - executes ... before printing whatever it says
\\
Long list: \noncurs
\vspace{-11pt}
\begin{multicols}{2} \noindent
  \code{\allsectionsfont{...}} \\
  \code{\partfont{... }}\\
  \code{\chapterfont{...}}\\
  \code{\sectionfont{...}}\\
  \code{\subsectionfont{...}}\\
  \code{\subsubsectionfont{...}}\\
\columnbreak
  \code{\paragraphfont{...}}\\
  \code{\subparagraphfont{...}}\\
  \code{\minisecfont{...}}\\
  \code{\partnumberfont{...}}\\
  \code{\parttitlefont{...}}\\
  \code{\chapternumberfont{...}}\\
  \code{\chaptertitlefont{...}}
\end{multicols}

\subsection{Dots}
\package{tocloft}\\
\code{\renewcommand{\cftchapleader}{\cftdotfill{\cftdotsep}}}

If we are in \article, we use:\\
\code{\renewcommand{\cftsecleader}{\cftdotfill{\cftdotsep}}}

\disclamer{if this command is misspelled the error doesn't appear immediately,
but when a chapter is added to the Table of Contents.}

\subsection{Colors}
\begin{latex}
% requires \usepackage{hyperref}
\hypersetup{colorlinks=true, linkcolor=cyan, citecolor=green,
  filecolor=black, urlcolor=blue}
\end{latex}

Define colors:\\
\code{\definecolor{name}{HTML}{RRGGBB}}, we can could also use \code{rgb} or \code{gray} instead of
\code{HTML}.

The option \texttt{[dvipsnames]} from \code{xcolor} defines PascalCase color names like in css like
'\code{MidnightBlue}'.

Using colors:
\codeshow{\colorbox{blue}{fundal}}\\
\codeshow{\fcolorbox{Black}{White}{text}} - \code{\fcolorbox{margine}{fundal}{text}}\\
\code{\pagecolor{White}}


\section{Romanian Names and diacritics}
\subsection[Quick way]{Quick way \noncurs}

\code{\usepackage[romanian]{babel}}

\subsection{Long way}
\begin{latex}
\renewcommand{\contentsname}{Cuprins}
\renewcommand{\chaptername}{Capitolul}
\renewcommand{\bibname}{B\lowercase{ibliografie}}
\renewcommand{\appendixname}{Anexa}
\renewcommand{\indexname}{I\lowercase{ndice}}
\renewcommand{\abstractname}{Rezumatul lucr\u{a}rii}
\end{latex}

\disclamer{don't combine The Long way with The Quick way.}

\subsectionNoncurs{Proper diacritics}
\package{combelow}\\
\codeshow{\cb{s} \cb{t}, not \c{s} \c{t}}\\
Easier diacritics:
\begin{latex}
\code{\usepackage[romanian]{babel}}
\useshorthands{'}
\defineshorthand{'t}{\cb{t}}
\end{latex}

\section[New Commands]{New Commands \noncurs}
\begin{latex}
\newcommand{\mat}[1]{\mathcal{M}_{#1}(\mathbb{R})}
\renewcommand{\contentsname}{Cuprins}

\usepackage{xparse}
% O, o optional, m= mandatory
\DeclareDocumentEnvironment{problema}{O {1} o m}{
  \begin{enumerate}[leftmargin=*]
    \addtocounter{enumi}{#1}
    \item
}{
  \end{enumerate}
  \IfValueT{#2}{\cppcode{#2}}
}
\newenvironment{\name}{begin}{end}

\end{latex}

\section{Fancy Headers}
  \subsection{Main Commands}

\begin{latex}
% inside the document
\pagestyle{fancy}
\thispagestyle{fancy} % only for one page
\lhead{\nouppercase{\leftmark}}
\chead{\chaptername}
\rhead{\rightmark}
\lfoot{\thechapter}
\cfoot{\thepage}
\rfoot{\thesection}

\fancyhf{} % reset the fancy settings
\end{latex}

%\thispagestyle{fancy}
\pagestyle{fancy}
\lhead{lm - \nouppercase{\leftmark}}
\chead{cn - \chaptername}
\rhead{rm - \nouppercase{\rightmark}}
\lfoot{ch - \thechapter}
\cfoot{pg - \thepage}
\rfoot{ts - \thesection}

\subsection{Page Styles}

\begin{longtable}{l l}
\justexplain{empty}{Empty headers and footers}
\justexplain{plain}{The default, just the page number}
\justexplain{myheadings}{The page number in header - right on even pages and left on odd pages}
\end{longtable}

\subsection{Information commands}
\begin{longtable}{l l}
  \justexplain{\thepage}{the current page number}
  \justexplain{\thechapter}{the number of the current chapter}
  \justexplain{\thesection}{the number of the current section}
  \justexplain{\chaptername}{the word chapter or equivalent in the current language}
  \justexplain{\leftmark}{the name\& number of the current level I structure \\& (Chapter for reports and books classes; Section for articles) in uppercase letters.}
  \justexplain{\rightmark}{the name and number of the current next to top-level structure\\& (Section for reports and books; Subsection for articles) in uppercase letters. }
\end{longtable}

OIDA

\section{Text alignment}

%\begin{center} \code{\begin{center} ... \end{center}}  \end{center}
%\begin{flushleft} \code{\begin{flushleft} ... \end{flushleft}} \end{flushleft}
%\begin{flushright} \code{\begin{flushright} ... \end{flushright}} \end{flushright}

\code{\begin{center} Centered \end{center}} \\
\code{\begin{flushleft} Left \end{flushleft}} \\
\code{\begin{flushright} Right \end{flushright}}

\chapter{The rest}
\pagestyle{plain}
\section{Page Size and layout}
\subsection[Quick way]{Quick way \noncurs}
\package{geometry}\\
\code{\geometry{a4paper,left=30mm,right=20mm,top=20mm,bottom=30mm}}

\subsection{Other way}
OIDA



\section{Symbols}
\subsection{The bane of students}
\begin{longtable}{l c l}
\code{$\backslash$ \textbackslash} & \textbackslash & backslash \\
\explain{\LaTeX}{\textit{that} symbol}
\end{longtable}

\subsection{Diacritics}
\begin{longtable}{l c l}
\explain{\^{a}}{circumflex}
\explain{\i \j}{dotless i and j}
\explain{\`{o}}{grave accent}
\explain{\'{o}}{acute accent}
\explain{\^{o}}{circumflex}
\explain{\"{o}}{umlaut, trema or dieresis}
\explain{\H{o}}{long Hungarian umlaut (double acute)}
\explain{\~{o}}{tilde}
\explain{\c{c}}{cedilla}
\explain{\k{a}}{ogonek}
\explain{\l{}}{barred l (l with stroke)}
\explain{\={o}}{macron accent (a bar over the letter)}
\explain{\b{o}}{bar under the letter}
\explain{\.{o}}{dot over the letter}
\explain{\d{u}}{dot under the letter}
\explain{\r{a}}{ring over the letter}
\explain{\u{o}}{breve over the letter}
\explain{\v{s}}{caron/háček}
\explain{\t{oo}}{"tie" (inverted u) over the two letters}
\explain{\o}{slashed o (o with stroke)}
\explain{\ss}{scharfes S}
\end{longtable}

\section{Spacing}
\subsection[Vertical]{Vertical (that way $\downarrow$)}
\indent

\code{\par} - Next paragraph (like  \code{\\} but with indent)

\code{\newpage \pagebreak \clearpage} - duh

\code{\smallskip \medskip \bigskip } - duh

\code{\cleardoublepage} - in book mode goto next odd page

\code{\\[10cm] \vspace{10pt}} - vertical space of height (might require newline after)

\subsection[Horizontal]{Horizontal (that way $\rightarrow$)}
\indent

\code{\-} - hyphenation point\\
\codeeg{Panzer\-kampf\-wagen VI Tiger Aus\-füh\-rung B ,,Königs\-tiger''}\\
aternative: \code{\hyphenation{Panzer-kampf-wagen}}
\smallskip

\code{\hspace{10pt}} - exact unit - might be negative
\code{M\hspace{-9pt}M} - M\hspace{-9pt}M

\code{\!} - negative space

\code{\thinspace \medspace \thickspace} $\iff$ \code{\, \: \;} - duh

\code{~ \ \quad \qquad} - bigger\\
\codeeg{a\, b\: c\; d~e\ f\quad g\qquad h}\\
Note: \code{~} is non breaking space.

\subsectionNoncurs{Line spacing}
\begin{latex}
\linespread{1.3} % ~1.5 
\linespread{1.6} % ~2 
\setstretch{2} % 2
\end{latex}
sau:
\begin{latex}
\singlespacing
\doublespacing

\begin{singlespace}  ...\end{singlespace}
\begin{doublespace}  ...\end{doublespace}
\begin{onehalfspace}  ...\end{onehalfspace}
\begin{spacing}{factor}  ...\end{spacing}{factor}

\end{latex}

\section[Miscellaneous]{\st{Pointless} Miscellaneous}
\subsection{Striketrough}
\package{soul}
\codeshow{\st{text}}
\subsection{Units}
\begin{itemize}\setlength\itemsep{-0.5em}
\item pt - a point is 1/72.27 inch, that means about 0.351 mm in non freedom units
\item mm - millimeter
\item cm - centimeter
\item in - inch
\item ex - roughly the height of an "x" in the current font
\item em - roughly the width of an  "M" (uppercase) in the current font
\end{itemize}

\subsection{Page Numbers}
\begin{longtable}{l l}
  \justexplain{\thispagestyle{empty}}{no page number}
  \justexplain{\pagenumbering{roman}}{roman i ii}
  \justexplain{\pagenumbering{Roman}}{Roman I II}
  \justexplain{\pagenumbering{arabic}}{arabic}
  \justexplain{\pagenumbering{alph}}{alph - a b }
  \justexplain{\pagenumbering{Alph}}{alph - A B }
\end{longtable}

\code{\pagenumbering} resets the counter,
use \code{\setcounter{page}{7}} afterwards to set the counter

\subsection{Self explanatory}
\begin{latex}
\include{smth.txt}
\tableofcontents

\newpage
\clearpage
\end{latex}

\section{Document Structure}
\subsection{Main structure}
\begin{latex}
\part{Part}
\chapter{C} % not found in article
\section{S}
\subsection{SS}
\subsubsection{SSS}
\paragraph{P}
\subparagraph{SP}
\end{latex}

Adding \code{[Short title]} will add that short name in the Table contents.

Adding a \texttt{*} will not add the structure to Table of contents and will not
number it.

To add it to the Table of contets use:
\code{\addcontentsline{toc}{section}{My name}}

Not adding this before a link or reference might not center the page properly\\
\code{\phantomsection}\\
e.g. \code{\phantomsection\label{Label}}


\subsection[Title page]{Title page\footnote{see the actual title page}}
\begin{latex}
\title{\LaTeX CheatSheet}
\author{Łucasz Zieliński\thanks{Dzięki światu}}
\date{\today}

\begin{titlepage} 
  \maketitle 
\end{titlepage}
\end{latex}


\subsection{Bibliography, abstract, index and appendix}
\begin{latex}
\appendix
\chapter{something}

\begin{abstract}
Bla bla
\end{abstract}
\end{latex}
\subsection{Bibliography}
\begin{latex}
\begin{thebibliography}{99}
    \bibitem{author/19} The Author, A book, 2018
\end{thebibliography}
\end{latex}
Cite from bibliography\cite{datta/17}.

\subsubsection{Index}
\begin{latex}
%in preamble
\makeindex
%in doc
\phantomsection
\addcontentsline{toc}{chapter}{Index} %index is not added by default to tableofcontents
\printindex
\end{latex}

Add word to index: \code{\index{Entry}}\index{Entry}

\section{Font styles}
\subsection{Text mode}
\begin{longtable}{l | l | l}
  \explain{\textnormal{Normal}}{The default font}
  \explain{\textrm{Roman}}{Roman font}
  \explain{\textsf{Sans}}{Sans serif}
  \explain{\texttt{Typewriter}}{Teletype - monospace font}
\hline
  \explain{\emph{Emph \emph{nested}}} {Emphasis. Might be nested}
  \explain{\textbf{Bold}} {Boldface}
  \explain{\textmd{Medium}}{Medium - default}
\hline
  \explain{\textup{Upright}}{Upright shape - default}
  \explain{\textit{abcx - Italics}}{Italic shape}
  \explain{\textsl{abcx - Slanted}}{Slanted shape}
\hline
  \explain{\textsc{Small Caps}}{Small Capitals}
  \explain{\uppercase{Uppercase $F=x$}}{Uppercase}
  \explain{\lowercase{Lowercase $F=x$}}{Lowercase}
\hline
  \explain{\tiny{tiny}}{ tiny}
  \explain{\scriptsize{script}}{scriptsize}
  \explain{\footnotesize{footnote}}{footnote}
  \explain{\small{small}}{small}
  \explain{\normalsize{normalsize}}{normal - default}
  \explain{\large{large}}{large}
  \explain{\Large{Large}}{Large}
  \explain{\LARGE{LARGE}}{LARGE}
  \explain{\huge{huge}}{huge}
  \explain{\Huge{Huge}}{Huge}
\end{longtable}

\subsection{Math mode}

\begin{longtable}{l | l | l}
  
  \explainMFont{\text}{Enter text mode}
  \explainMFont{\mathnormal}{Default}
  \explainMFont{\mathrm}{Roman - one word functions}
  \explainMFont{\mathit}{Italics - words spaced more naturally}
  \explainMFont{\mathbf}{Boldface}
  \explainMFont{\mathsf}{Sans serif}
  \explainMFont{\mathtt}{Teletype - monospaced}
  \explainMFont{\mathfrak}{Fraktur}
  \explainMFontUpper{\mathcal}{Caligraphy - of note $\mathcal{M}$atrix}
  \explainMFontUpper{\mathbb}{Blackboard bold - of note $\mathbb{R}$}
  \explainMFontUpper{\mathscr}{Script - Super  $\mathscr{F}$\!{}ancy}
\end{longtable}

%\item{

\clearpage
\phantomsection
\addcontentsline{toc}{chapter}{Index}

\printindex

\begin{thebibliography}{99}
    \bibitem{datta/17} Dilip Datta,\textit{\LaTeX{} in $24$ Hours. A Practical Guide for Scientific Writting}, Springer, Cham,2017

\end{thebibliography}
\end{document}

Left at chapter 4, lab 5&6
