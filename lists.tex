\section{Lists}

There are 3 main environments to use lists: \coden{enumerate}, \coden{itemize}
and \coden{description}.

New items are inserted with \coden{\item[sym]}. 

There's a default vertical space between items.
This space can be modified by writing \code{\setlength{\itemsep}{5mm}} after \code{\begin{env}}.

We can nest those a maximum of 4 times.

\begin{example}
\begin{enumerate}
  \item Operating systems
  \begin{enumerate}\setlength{\itemsep}{0mm}
  \item[J.] Linux
  \item Windows
  \item DOS
  \item BSD
  \end{enumerate}
\item Programing languages
  \begin{description}
    \item[1] \texttt{C++}
    \item[b] \texttt{D}
    \item[III] \texttt{F\#}
  \end{description}
\end{enumerate}
\end{example}


\subsection{\texttt{enumerate}}
Items can be labeled (the other 2 can't).

This redefines the labels names:
\begin{latex}
\renewcommand{\labelenumi}{\Roman{enumi}.}
\renewcommand{\labelenumii}{(\arabic{enumii})}
\renewcommand{\labelenumiii}{(\alph{enumiii})}
\renewcommand{\labelenumiv}{(\roman{enumiv})}
\end{latex}

We can redefine the numbering using:
\begin{latex}
\renewcommand{\theenumi}{\Roman{enumi}.}
\renewcommand{\theenumii}{(\arabic{enumii})}
\renewcommand{\theenumiii}{(\alph{enumiii})}
\renewcommand{\theenumiv}{(\roman{enumiv})}
\end{latex}

\begin{example}
\renewcommand{\theenumi}{Point \Alph{enumi}}
\renewcommand{\labelenumi}{I \Alph{enumi}}
\begin{enumerate}
\item First\label{first}
\item Second
\end{enumerate}
We reference an item \ref{first}
\end{example}

If we use the package \coden{enumitem} we can change the indent:

\begin{example}
\begin{enumerate}
\item Level 1
  \begin{enumerate}[leftmargin=0cm]
    \item Level 2
    \end{enumerate}
\item Also Level 1
  \begin{enumerate}
    \item Also Level 2
  \end{enumerate}
\end{enumerate}
\end{example}

Using \coden{enumitem} also allows us to have two separate lists with continuous
numbering:\\
\begin{example}
\begin{enumerate}
\item First
\end{enumerate}
Some text
\begin{enumerate}[resume]
\item Second
\end{enumerate}
\end{example}

\subsection{\texttt{itemize}}
\texttt{itemize} can also have it's symbols changed:\\
\begin{example}
  \renewcommand{\labelitemi}{$\bigstar$}
  \renewcommand{\labelitemii}{$\checkmark$}
  \renewcommand{\labelitemiii}{$\sharp$}
  \renewcommand{\labelitemiv}{$\maltese$}
  \begin{itemize}
    \item Foo
\end{itemize}
\end{example}
